\documentclass[a4paper,10pt]{article}

\usepackage[utf8x]{inputenc}
\usepackage[francais,french]{babel}
\usepackage{fontenc}
\usepackage{graphicx}
\usepackage{url}

\usepackage[dvips]{hyperref}
\usepackage{amssymb}

\newcommand{\coquille}{\textsc{Coquille}}
\newcommand{\coqcode}[1]{\texttt{#1}}

\date{10 novembre 2009}
\title{Rapport final du projet \coquille{}}
\author{M1 Informatique Fondamentale - ENS Lyon}

\begin{document}

\maketitle
\newpage
\setcounter{section}{1}
\section*{Introduction}

\subsection{Présentation générale}

L'objectif de \coquille{} (\emph{Coq User-Interactive Library Learning Expert}) est la réalisation d'un environnement de démonstrations de preuves mathématiques proche du monde des classes préparatoires. Il doit être élaboré à partir de l'assistant de preuves Coq. L'objectif est en somme d'élargir la portée de Coq au mathématiques de classes préparatoires, et à des mathématiciens non forcément experts en calcul des constructions et en théorie des types.

\subsection{Motivation}

La preuve assistée par ordinateur ouvre aux mathématiques de nouvelles perspectives. L'exemple le plus connu est la démonstration en Coq du théorème des quatre couleurs. Ce théorème n'a jamais été démontré sans l'usage de l'ordinateur. Même si pour l'instant, il y a très peu de théorèmes à avoir été démontrés d'abord en Coq, on peut espérer qu'en améliorant le langage de preuve, les bibliothèques de théorèmes et de tactiques, et en rendant Coq plus facile à utiliser, démontrer un théorème sera plus facile en Coq. Cela révolutionnera la façon de faire des mathématiques. Nous voulons apporter notre pierre à cet édifice.
%% Faut arrêter la Marie-Jeanne les enfants.

En classes préparatoires, les élèves sont confrontés à un grand nombre de problèmes souvent répétitifs et non triviaux qui n'ont pas toujours de corrigés. Avoir un logiciel qui résoudrait le problème et donnerait sa démonstration permettrait d'avoir une correction automatique. Nous voulons réaliser un tel logiciel.

\subsection{Public visé}

Nous souhaitons avant tout fournir un outil utilisable dans le contexte des classes préparatoires, mais il serait très fortement envisageable de le diffuser dans le monde de l'industrie et de la recherche (par exemple pour aider à confirmer des conjectures ou même des preuves lourdes en calculs et apporter une surcouche de fiabilité aux chercheurs)

\subsection{Approche}

La résolution devra être la plus automatique possible. Mais les problèmes pouvant être difficiles, il n'est pas exclu que l'ordinateur demande de l'aide à l'utilisateur par moments.

Le logiciel doit être ergonomique : même si le public visé est assez scientifique pour vouloir prouver un théorème de mathématiques, il n'est pas forcément informaticien.

Nous souhaitons proposer à l'utilisateur un certain nombre de tactiques de haut niveau, correspondant aux différents chapitres du programme des classes préparatoires. Il sera nécessaire de mettre aussi à disposition des bibliothèques de preuves pour chaque chapitre. Il y aura toujours deux versions pour chaque chapitre :
\begin{itemize}
  \item la version que l'on aura directement codée, stable.
  \item la version issue de l'apprentissage, qui continuera de s'améliorer au fur et à mesure que l'usager utilisera le logiciel.
\end{itemize}

Pour parvenir à ce résultat, ce projet de recherche s'est constitué autour de deux parties :

\begin{enumerate}
\item un aspect scientifique dans lequel les bibliothèques de coq sont élargies aux théorèmes et définitions nécessaires aux mathématiques de classes préparatoires et dans lequel des méthodes de résolution et d'apprentissage de résolution sont proposées ;
\item un second aspect du projet est appliqué à la diffusion et à la mise à disposition de ces bibliothèques et fonctionnalités, à travers un IDE ergonomique et efficace.
\end{enumerate}

\subsection{Ressources humaines}

Ce projet est développé par 22 élèves du M1 d'informatique fondamentale de l'ENS Lyon. Les tâches se sont réparties entre quatre groupes de travail (WP) dont nous allons décrire l'avancement à mi-parcours. Liste des participants et découpage des groupes de travail est en annexe. Les WP se réunissent toutes les semaines : une heure séparés ainsi qu'une heure de mise en commun.

\section{Avancement des groupes de travail}

\subsection{WP communication}

Ce groupe est dirigé par Mathieu Barthélémy, et compte 5 participants.

\subsubsection{Objectifs}
L'objectif de ce groupe de travail est de faciliter la coordination des autres groupes de travail et de promouvoir le projet et ses résultats auprès de potentiels utilisateurs.

\subsubsection{Matériels et méthodes}
Pour atteindre ces objectifs ce groupe de travail a réalisé un site internet~\cite{coquille}. Ce site rassemble : un forum et un wiki~\cite{wikicoquille} pour faciliter la communication interne, et une présentation du projet, une page de contacts et les documents relatifs à \coquille{} pour promouvoir le projet et ses résultats auprès de ses utilisateurs. Ce groupe a par ailleurs réalisé la rédaction des différents rapports hebdomadaires et coordonné la rédaction du rapport de mi-projet. Enfin, des documents de communications internes et des formations HTML et \LaTeX{} ont par ailleurs été réalisées.

\subsubsection{Perspectives}
Ce groupe de travail prévoit de rendre la documentation disponible sur le site internet, pour tenir compte de l'avancée des WP. Il prévoit aussi de réaliser le rapport final et de préparer une démonstration du projet. Il est prévu de rendre le site plus interactif en permettant la mise à jour automatique de la documentation. Il est prévu de faire connaître \coquille{} en dialoguant sur des forums ou avec des utilisateurs potentiels.

\subsubsection{Résultats}
Les principales réalisations de ce groupe de travail sont le site internet~\cite{coquille} et le présent rapport.

\subsection{WP preuves et tactiques}
À l'origine séparés, le groupe preuve a phagocyté le groupe tactiques au vu de la proximité de leurs tâches et de leurs équipes. La partie preuves est dirigée par Pierre-Marie Pédrot et la partie tactique par Marthe Bonamy, pour un total de 9 participants théoriques (dont quelques égarés).

\subsubsection{Objectifs}
Le but de ce groupe de travail est d'établir une liste de théorèmes vus en classes préparatoires dans les domaines suivants : l'algèbre, l'analyse, la topologie et l'arithmétique. Dans un deuxième temps l'objectif est de les implémenter en Coq. Enfin le dernier objectif est de proposer des tactiques évoluées d'aide aux utilisateurs de Coq.

\subsubsection{État de l'art}
Plusieurs domaines mathématiques ont été formalisés en Coq~\cite{ccorn, stdlib}, mais relativement succinctement, et la quasi-totalité des formules que l'on voit dans les cours de classes préparatoires en sont absentes.

\subsubsection{Matériels et méthodes}
D'un côté, certains membres du groupe ont compilé des livres de classes préparatoires, avec des résultats qui se sont révélés par la suite inutilisés. Une autre partie du groupe est en train d'écrire les preuves en Coq. Afin d'être fidèle aux démonstrations attendues, la logique classique est utilisée au besoin. Nous avons toutefois essayé de limiter au maximum son utilisation.

Pour ne pas reproduire les défauts de la bibliothèque standard, nous nous sommes entendus pour suivre à la lettre les conventions de nommages recommandées dans la proposition de Hugo Herbelin, même si cela s'avère parfois fastidieux.

% Pour la partie tactiques, le langage utilisé est évidemment ltac.

\subsubsection{Résultats et enjeux}

Des modules seront bientôt disponibles dans les domaines suivants : arith\-métique, nombres complexes, nombres réels, théorie des ensembles, et topologie.

\paragraph{Arithmétique}

L'un des résultats importants de l'arithmétique est la démon\-stration du petit théorème de Fermat. Il a fallu pour ce faire prouver une série de lemmes sur la divisibilité et les nombres premiers : propriétés élémentaires, notamment le théorème de Gauss.

D'autres résultats fondamentaux ont été abordés : définition des coefficients binomiaux de Pascal sous forme inductive et équivalence avec la forme factorielle, et par suite, formule de Newton.

Il faut aussi citer une série de lemmes d'arithmétiques intermédiaires qui manquent cruellement dans la bibliothèque standard (produit fini, justification de la réindexation, stabilité par produit, $0$ absorbant, etc.), de même que d'outils de manipulation des entiers (définition de la double induction, triple induction, induction forte) ainsi que d'outils pour construire facilement d'autres principes d'induction sur les entiers.

\paragraph{Topologie}

L'approche standard de définitions des objets mathématiques dans Coq est basé sur la définition constructive d'éléments moins généraux vers d'autres plus généraux. Les principes constructifs justifient cette approche. Cependant certains objets ne peuvent être construits, par exemple dans \coqcode{Reals}, le fait que $\mathbb{R}$ est un corps n'est pas une propriété mais un axiome.

Ainsi pour faciliter la définition d'objets plus généraux, on peut tenter de définir les objets les plus généraux dans un premier temps. Un besoin suggéré par la manipulation des suites, séries ou même des complexes est une homogénéisation des tactiques pour simplifier les expressions qui supportent des propriétés de groupe, d'anneau ou encore d'espace vectoriel. On cherche à géné\-raliser plus généralement la notion d'espace.

Ont été définis les espaces topologiques, métriques et vectoriels. Il reste encore beaucoup de travail, et l'utilisation des \emph{typeclasses} semble incontournable.

\paragraph{Suites réelles}

Les suites sont parmi les outils de base de l'analyse réelle, et le moins qu'on puisse dire, c'est que la bibliothèque standard de Coq n'est ni très étendue dans ce domaine, ni très cohérente\footnote{Comme à peu près tout \coqcode{stdlib}.}. Il y a donc une masse de travail non-négligeable. Les résultats sont très simples et pourtant fondamentaux.

Nous disposons d'une partie entière dédiée à la convergence (au sens large) des suites, dont la compatibilité avec de nombreuses opérations. Dans la même veine, les relations de Landau ont été définies et de nombreuses propriétés génériques ont été démontrées.

D'autres outils de base ont été implémentés : suites usuelles et leurs propriétés, propriétés asymptotiques, propriétés des suites partielles, etc.

De nombreux résultats sont généralistes ; nous envisageons de les implé\-menter avec des \emph{typeclass} sur les espaces vectoriels par la suite.

\paragraph{Séries entières}

Les séries entières sont une partie intégrante du programme des classes préparatoires, mais elles sont pratiquement totalement absentes de la bibliothèque standard. Une partie importante de nos deux premières semaines de travail a été de trouver les définitions des concepts nous assurant une manipulation aisée de ceux-ci. Nous avons choisi de commencer par implémenter les théorèmes fondamentaux traitant des séries entières : lemme d'Abel, critère de convergence de d'Alembert, caractérisation du rayon de convergence et dériva\-bilité de la série sur son disque de convergence. C'est là que se situe la majeure partie du travail sur les séries entières.

Pour les développements futurs plusieurs pistes sont à explorer : démontrer le théorème des zéros isolés, écrire une fonction retournant la dérivée $n$-ième d'une série, étendre au cas complexe...

\paragraph{Complexes}

Grands absents de la bibliothèque standard, nous avons défini les complexes et démontrés certains théorèmes de base.

\paragraph{Logique}

On dispose de deux résultats importants dans cette section : indé\-nom\-brabilité de $\mathbb{R}$ et implications classiques de l'axiomatique des réels de Coq. Si le premier résultat est bien connu\footnote{Mais non encore démontré en Coq...}, le second est pour le moins un peu nébuleux et attire l'attention de certains coqueux.

\subsection{Objectifs}

Ce groupe compte créer un codicile à la bibliothèque standard, effectuer du renommage et documenter cette dernière, ainsi que $L_{tac}$. L'objectif serait d'y être intégré à terme.

\subsection{WP IDE}

Ce groupe est dirigé par Yann Hourdel et compte 6 participants.

\subsubsection{Objectifs}

Ce groupe a pour but de fournir une interface interactive, ergonomique, et conviviale, entre Coq, les bibliothèques disponibles, les tactiques disponibles ou fournies par le groupe preuve, et enfin les fonctionnalités liées à l'apprentissage sur lesquelles travaille le groupe apprentissage.

\subsubsection{État de l'art}

Coq est doté d'un environnement nommé coqIDE qui permet d'exécuter pas à pas une preuve et de récupérer les réponses de Coq et l'environnement à chaque étape. En revanche, CoqIDE ne nous semble pas particulièrement adapté pour ajouter des fonctionnalités comme l'apprentissage et la navigation dans les bibliothèques disponibles. Nous avons donc jugé qu'il serait plus efficace de se baser sur un nouvelle IDE, plus ergonomique.

Les fonctionnalités prévues dans notre'IDE sont entre autres :
   \begin{itemize}
   \item Complétion automatique ;
   \item Coloration syntaxique ;
   \item Documentation des fonctions ;
   \item Indentation automatique ;
   \item Version online du logiciel ;
   \item Gestion des bibliothèques de théorèmes de l'utilisateur ;
   \item Onglets ;
   \item Insertion automatique avec raccourcis claviers avec la possibilité d'en rajouter ;
   \item Gestion des commentaires des fonctions et création de documentation automatique ;
   \item Affichage graphique en \LaTeX{} des théorèmes ;
\end{itemize}

\subsubsection{Méthodes et approches}
Afin de déterminer le langage de cet IDE nous avons mis en compétition deux mini-IDEs, l'un codé en C++ avec Qt et l'autre en Java avec Swing. Il en est ressorti que le Qt était plus adapté.

\subsubsection{Résultats et perspectives}
L'IDE est désormais fonctionnel sous une forme minimale qui dialogue avec Coq, et résume les principales fonctionnalités de coqIDE.

% Qu'est-ce qu'on fait maintenant ?
% -> on commence à penser à intégrer ce que font les autres ? Comment on va pouvoir proposer des tactiques ou de l'apprentissage ?

\subsection{WP Apprentissage}

Ce groupe est dirigé par Guillaume Aupy et compte 6 participants.

\subsubsection{Objectifs}
L'objectif de ce groupe de travail est, à partir de nombreux exercices, de créer des tactiques spécifiques à des types d'exercices.

\subsubsection{Matériels et méthodes}

On utilise un arbre de décision. À chaque n\oe{}ud de l'arbre de décision se trouve une question portant sur les arbres syntaxiques des hypothèses et du but. Pour la première question, on ne s'autorise à interroger que les racines. Ensuite, on s'autorise à interroger les fils des arbres déjà interrogés. Les questions sont « quel est le token à la racine de l'arbre $x$ ? » et « est-ce que les arbres $x$ et $y$ ont le même token en racine ? ».

Pour choisir la question à poser, on calcule la question qui nous donne le meilleur rapport information apportée selon la définition de von Newmann sur nombre de réponses. On cherche donc à minimiser la taille de l'arbre (et non sa profondeur), car c'est d'elle que dépend le temps d'exécution et l'occupation de la mémoire.

\subsubsection{Difficultés}  %à reformuler
La grosse difficulté du parcours de l'arbre est la gestion des choix. En effet, comme il y a plusieurs hypothèses, lorsqu'on demande « quelle est la racine de l'hypothèse », on fait un choix. Pour plus d'efficacité, on repousse ce choix au plus loin : on passe plusieurs choix possibles. Les bonnes représentations sont essentielles car elles permettent d'éliminer les choix caduques, les choix dépendant d'autres choix.

Pour l'algorithme d'apprentissage, le problème est d'avoir une complexité correcte. Un algorithme naïf donnerait une complexité cubique par rapport à celle que nous avons actuellement. %à préciser

\subsubsection{Résultats}
Le parcours d'arbres et le rafinement des données sont utilisables (leur complexité est améliorable et des améliorations sont possibles). L'algorithme d'apprentissage a été réalisé.

\nocite{*}
\bibliography{biblio}{}
\bibliographystyle{acm}

% \begin{itemize}
% \item \url{http://perso.ens-lyon.fr/jeanmarie.madiot/coquille/forum}
% \item \url{http://coquille.wikispot.org/}
% \end{itemize}

%Les livres de classes préparatoires \\
%Les bibliothèques IDE :

\end{document}

