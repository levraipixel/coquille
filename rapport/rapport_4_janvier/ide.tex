\section{Contexte initial}

    \subsection{État de l'art : coqide}

        \subsubsection{Pas vraiment user-friendly}

            Ici on explique pourquoi on aime pas coder dans coqide.

        \subsubsection{Très opaque}

            Ici on explique qu'on a regardé le code droit dans les yeux et qu'on pouvait pas faire quelque chose de simple dessus.

    \subsection{Les besoins du projet Coquille}

        \subsubsection{WP Preuves / Tactiques}

            A priori aucune attente de ce WP

        \subsubsection{WP Apprentissage}

            Ici on parle des besoins de ce WP

    \subsection{Que faire...}

        \subsubsection{Tenter de modifier coqide ?}

            On explique pourquoi c'était pas envisageable

        \subsubsection{Créer des plugins pour les éditeurs classiques ?}

            Idem
            
        \subsubsection{Coder un IDE from scratch ?}
        
            Expliquer les avantages de cette solution, présenter quelques inconvénients mais pas trop. 

\section{Quelques choix à faire}

    \subsection{Le Langage utilisé}
    
        On parle un peu du C++
        
    \subsection{Les Bibliothèques utilisées}
    
        On parle de Qt, QCodeEdit, QTermWidget et des soucis qu'on a eu avec : bcp de temps passé à comprendre mais aussi bcp de choses apprises.
        
    \subsection{Dialogue avec Coq}
    
        On parle de Coqtop un peu, et du fait que ça paraît très sommaire.
        On parle des échanges de mails avec des responsables à l'inria qui nous disent que ya pas d'API.
        
\section{Les problèmes rencontrés}

    \subsection{Au niveau du langage}
    
        Rien, d'après moi...
        
    \subsection{Au niveau des bibliothèques}
    
        Détailler quelques trucs.
        On peut parler du problème de coloriage du texte qui nous oblige à ne pas colorier le tout dernier point.
        
    \subsection{Au niveau du dialogue avec Coq}
    
        Les problèmes principaux sont là. On peut parler des trucs qui nous ont longtemps posé problème mais qui sont contournés.
        Liste non exhaustive à détailler :
        - La discrimination des erreurs : StdErr pour le prompt uniquement. Absurde...
        - Les réponses sans réponse : Les \coqcode{Require}, \coqcode{Proof} et companie
        - Revenir en arrière : Un véritable casse-tête... Le seul PB non résolu actuellement
        
\section{Le résultat actuel}

    \subsection{Ce que Coquille fait et que CoqIde ne fait pas}
    
        \subsubsection{Au niveau du code}
        
            \begin{itemize}
                \item Les numéros de ligne.
                \item Le code folding (replier des lignes de code en une seule, pour une preuve par exemple).
                \item Des raccourcis plus "classiques", et personnalisables.
                \item Possibilité de faire "Redo" après des "Undo".
            \end{itemize}
            
        \subsubsection{Au niveau du langage}

            \begin{itemize}
                \item La gestion de Ltac Debug, avec une interface de parcours des résultats
                \item Plusieurs instances de Coqtop, une par onglet ouvert
                \item Un affichage des résultats au choix : classique ou $\LaTeX$-like
                \item L'action « Next/Previous » d'envoi d'une commande est considérée comme une action comme les autres, donc "Undo/Redo" agit dessus
            \end{itemize}
            
    \subsection{Ce que CoqIde fait et que Coquille ne fait pas (encore)}
    
        \subsubsection{Au niveau du code}

            \begin{itemize}
                \item Lister les actions disponibles par click droit sur une hypothèse ou un but.
            \end{itemize}

        \subsubsection{Au niveau du langage}
        
            \begin{itemize}
                \item Le Proof Wizard.
                \item La gestion des Write State / Restore State
                \item La gestion de l'aide.
                \item La gestions de la compilation.
            \end{itemize}

\section{Les objectifs}

Ici on détaille ce qu'on voudrait faire après la fin du projet.
