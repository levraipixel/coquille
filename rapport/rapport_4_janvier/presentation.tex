\section{Introduction}

\subsection{Présentation générale}

L'objectif de \coquille{} (\emph{Coq User-Interactive Library Learning Expert}) est la réalisation d'un environnement de preuve assistée par ordinateur dotant l'utilisateur des outils nécessaires à l'implémentation de résultats proches de ceux démontrés dans les classes préparatoires. Il doit être élaboré à partir de l'assistant de preuves \coq{}.

L'objectif est en somme d'élargir la portée de \coq{} aux mathématiques de classes préparatoires, et à des mathématiciens non forcément experts en calcul des constructions et en théorie des types.

\subsection{Motivation}

La preuve assistée par ordinateur ouvre de nouvelles perspectives pour les mathématiques modernes. Le chemin parcouru depuis \texttt{Automath}\cite{automath} (fin des années 60) permet de parler réellement de preuve interactive : l'utilisateur n'a plus à fournir un $\lambda$-terme du bon type mais peut le construire \emph{via} l'utilisation de tactiques et de lemmes définis dans la bibliothèque standard ou par ses soins.

La principale limite des assistants de preuves n'est donc plus la complexité de la théorie sous-jacente\footnote{Il est tout à fait possible de démontrer de nombreux théorèmes sans comprendre toutes les subtilités du calcul des constructions.} mais l'absence de certaines bibliothèques de base (une bonne partie du programme des classes préparatoires n'est pas couvert), la nécessité de démontrer de nombreux sous-buts (relativement) triviaux ainsi que l'austérité des interfaces (ce qui peut rebuter les non-informaticiens).

Partant de ces constatations, le projet \coquille{} tâchera de développer des outils permettant de démontrer simplement des théorèmes énoncés en classes préparatoires.

\subsection{Public visé}

Nous souhaitons avant tout fournir un outil utilisable dans le contexte des classes préparatoires, mais il serait très fortement envisageable de le diffuser dans le monde de l'industrie et de la recherche (par exemple pour aider à confirmer des conjectures ou même des preuves lourdes en calculs et apporter une surcouche de fiabilité aux chercheurs).

\subsection{Approches}

Le projet \coquille{} a trois composantes principales et chacune a pour but de combler une partie des limites évoquées précédemment.

\paragraph{Ergonomie} Le logiciel doit être ergonomique : même si le public visé est assez scientifique pour vouloir prouver un théorème de mathématiques, il n'est pas forcément informaticien. Le projet \coquille{} développera donc une interface intuitive et fonctionnelle pour Coq.

\paragraph{Simplicité} La démonstration de lemmes devra être la plus simple possible, nous tâcherons donc d'automatiser au maximum le processus. Les problèmes pouvant être difficiles (indécidables par exemple), il n'est bien entendu pas exclu que l'ordinateur demande de l'aide à l'utilisateur.

\paragraph{Rapidité} Nous souhaitons proposer à l'utilisateur un certain nombre de bibliothèques et de tactiques de haut niveau correspondant aux différents chapitres du programme des classes préparatoires. L'existence de ces outils permettra d'implémenter rapidement des preuves complexes en réutilisant des résultats déjà existants.

\bigskip

Pour parvenir à ce résultat, ce projet de recherche s'est organisé autour de trois axes.

\paragraph{Résultats mathématiques majeurs} Les groupes de travail \emph{Preuves} et \emph{Tactiques} développent des bibliothèques regroupant les concepts évoqués en classes préparatoires ainsi que les résultats majeurs les concernant.

\paragraph{Démonstrations automatiques} Le groupe de travail \emph{Apprentissage} a pour but de permettre la résolution automatique de résultats simples \emph{via} l'adaptation d'un algorithme d'apprentissage classique au contexte d'une démonstration en \coq{}.

\paragraph{Prise en main facile} Le groupe de travail \emph{IDE} développe une interface intuitive et ergonomique permettant une utilisation simple de \coq{} et de l'intégralité des résultats démontrés au sein du projet \coquille{}.

\subsection{Ressources humaines}

Ce projet est développé par 22 élèves du M1 d'informatique fondamentale de l'ENS Lyon. Les tâches se sont réparties entre quatre groupes de travail (WP) dont nous allons décrire les développements. La liste des participants et découpage des groupes de travail est en annexe.
